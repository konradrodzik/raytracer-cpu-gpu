
% *************** LEd - dokument oparty na szablonie pracy magisterskiej ***************

\documentclass[12pt,apalike,a4paper,openright,twoside,makeidx]{memoir}
\input{style.tex}
\usepackage[polish]{babel}

\begin{document}
% *************** Front matter ***************
\frontmatter
% *************** Front matter ***************

% ************************************************************
% W tym miejscu mo�esz zdefiniowa� wygl�d strony tytu�owej
% Mo�esz tak�e zdefiniowa� dedykacj� albo wy��czy� j�
% ************************************************************

% *************** Strona tytu�owa ***************
\pagestyle{empty}
\sffamily

\linespread{1.0}
\noindent



\vfill\vfill
\begin{center}
    \Large\bfseries
    PRACA DYPLOMOWA - IN�YNIERSKA\\
\end{center}

\vfill
\begin{center}
    \Large
    Konrad Rodzik\\
    nr albumu: 215039
\end{center}

\vfill
\begin{center}
    \Large\bfseries
    NVIDIA CUDA jako technologia przy�pieszaj�ca metode �ledzenia promieni
\end{center}

\vfill\vfill\vfill
\begin{flushright}
    \Large
    Kierownik pracy:\\ Prof. nazw. dr hab. in�. Dariusz Sawicki
\end{flushright}
\vfill

\begin{flushleft}
....................................................\\
\ \ ocena pracy \\
\vfill
....................................................\\
\ \ data i podpis Przewodnicz�cego\\
\ \ Komisji Egzaminacyjnej Dyplomowego
\end{flushleft}

\vfill
\begin{center}
\large
    Warszawa, 2011
\end{center}





% *************** Koniec front matter ***************




\definecolor{ListingBackground}{rgb}{0.95,0.95,0.95}


\lstdefinestyle{outcode}
{
basicstyle={\footnotesize},
keywordstyle=\color[rgb]{0,0,1}, 
commentstyle=\color[rgb]{0.133,0.545,0.133}, 
stringstyle=\color[rgb]{0.627,0.126,0.941}, 
numbers=left, 
stepnumber=1, 
firstnumber=1,
numberfirstline=true,
numberblanklines=false, 
numbersep=10pt, 
tabsize=2,
xleftmargin=17pt,
framexleftmargin=3pt,
framexbottommargin=2pt,
framextopmargin=2pt,
framexrightmargin=0pt,
showstringspaces=true,
backgroundcolor={\color{ListingBackground}},
extendedchars=true,
% title=\lstname,
captionpos=b,
% abovecaptionskip=1pt,
% belowcaptionskip=1pt,
frame=tb,
framerule=0.1pt, 
}







% *************** Main matter ***************
\mainmatter
% ********** Podzi�kowania **********
\clearpage
\chapter*{Podzi�kowania}

Chcia�bym bardzo podzi�kowa� moim rodzicom oraz mojej dziewczynie. Bez ich wsparcia i bod�c�w motywacyjnych powstanie tej pracy nie by�o by mo�liwe.

Na ko�cu lecz nie mniej wa�ne podziekowania dla mojego sprz�tu, kt�ry du�o wycierpia� podczas postawania niniejszej pracy. Karta graficzna wymieniana byla 3 razy...
 

% ********** Koniec podzi�kowa� **********

% ********** Rozdzia� 1 **********
\chapter{Wst�p}
\label{sec:chapter1}


\section{Wprowadzenie}
\label{sec:chapter1:Wprowadzenie}
Raytracing jest technik� s�u��c� do generowania foto realistycznych obraz�w scen 3D. Na przestrzeni lat technika ta ci�gle si� rozwija�a. Doczeka�a si� wielu modyfikacji, kt�re usprawniaj� proces generowania realistycznej grafiki. Takimi technikami mog� by� mi�dzy innymi PathTracing, PhotonMapping, Radiostity i wiele innych. Z dnia na dzie� wykorzystywanie raytracingu ci�gle ro�nie. 
W dzisiejszych czasach w grafice komputerowej oraz w kinematografii do uzyskania realistycznych efekt�w u�ywana jest metoda �ledzenia promieni. Dzi�ki takim zabiegom jeste�my w stanie dos�ownie zasymulowa� sceny oraz zjawiska, kt�re nie musz� istnie� w rzeczywistym �wiecie. Czas generowania pojedynczej klatki/uj�cia takiej sceny niekiedy potrafi by� liczony nawet w godzinach. Dlatego technika ta nie doczeka�a si� jeszcze swojej wielkiej chwili w przemy�le rozrywkowym jakim s� np. gry komputerowe oraz inne aplikacje generuj�ce grafik� 3D w czasie rzeczywistym.

\begin{figure}[h]
\begin{center}
\begin{minipage}[b]{4cm}
\centering
\includegraphics[width=\textwidth]{roz1/img/tron.jpg}\\\textit{a) Tron Legacy}
\end{minipage}
\begin{minipage}[b]{4cm}
\centering
\includegraphics[width=\textwidth]{roz1/img/avatar.jpg}\\\textit{b) Avatar}
\end{minipage}
\begin{minipage}[b]{4cm}
\centering
\includegraphics[width=\textwidth]{roz1/img/shreck.jpg}\\\textit{c) Shrek}
\end{minipage}
\caption{Uj�cia filmowe stworzone za pomoc� technik komputerowych}
\label{fig:filmy_rt}
\end{center}
\end{figure}


\section{Motywacja}
\label{sec:chapter1:Motywacja}
G��wnym bod�cem motywacyjnym do napisania tej pracy by�a ch�� poszerzenia dotychczasowej wiedzy na temat  przy�pieszania oblicze� przy pomocy nowej technologii NVIDIA CUDA. Dodatkowymi czynnikami motywacyjnymi by�o zami�owanie do grafiki komputerowej oraz do tworzenia aplikacji oblicze� czasu rzeczywistego. Na co dzie� zajmuj� si� programowaniem gier/aplikacji na komputery klasy PC oraz urz�dzenia mobilne. Uwa�am, �e w przysz�o�ci przedstawiona przeze mnie w tej pracy metoda �ledzenia promieni b�dzie mia�a zastosowanie w grach oraz aplikacjach wykorzystuj�cych grafik� czasu rzeczywistego.


\section{Terminologia wykorzystywana w pracy}
\label{sec:chapter1:terminologia}
\begin{itemize} 
\item Developer - Tw�rca/programista aplikacji
\item CUDA - Technologia stworzona przez firm� NVIDIA w 2007 roku. Umo�liwia r�wnoleg�e obliczenia na mikroprocesorach karty graficznej
\item Raytracing - metoda generowania obrazu za pomoc� �ledzenia promieni.
\item Benchmark - Aplikacji testowa, kt�ra profiluje wydajno�� i zbiera informacje.
\item Warp - blok w�tk�w przydzielony na multiprocesor
\item DirectX - technologia graficzna firmy Microsoft. Umo�liwia wy�wietlanie wysokiej jako�ci grafiki 2D/3D.
\item Aliasing - Zdeformowany, o z�ej jako�ci obraz kt�ry powstaje podczas rastaryzacji, powodowany przez zbyt ma�a cz�stotliwo�� pr�bkowania na pojedy�czy pixel obrazu. Przeciwdzia�a si� temu efektowi poprzez antyaliasing oraz w raytracingu poprzez super-sampling
\item Super-sampling - spos�b na zwi�kszenie jako�ci generowanych scen. Polega na �ledzeniu wielu promieni �wietlnych na pojedy�czy pixel generowanego obrazu.
\end{itemize} 


\section{Dost�pne technologie, pozwalaj�ce zr�wnolegli� obliczenia na kartach graficznych}
\label{sec:chapter1:Dost�pneTechnologie}
Poni�ej zaprez�towane zosta�y trzy wybrane technologi� wspomagaj�ce r�wnoleg�e  obliczenia na kartach graficznych. Niemniej jednak badania przeprowadzone i opisane w dalszej cz�ci pracy b�d� skupia�y si� na wykorzystaniu jednej z tych metod, a mianowicie technologii NVIDIA CUDA.


\subsection{Open Computing Language (OpenCL)}
Technologia tak zainicjowana zosta�a przez firm� Apple. Do inicjatywy i rozwijania tej technologii w��czy�y si� w p�niejszym czasie inne firmy takie jak: AMD, IBM, Intel, NVIDIA. W roku 2008 sformowana zosta�a grupa Khronos skupiaj�ca powy�sze firmy oraz wiele innych nale��cych do bran�y IT. Grupa ta czuwa nad rozwojem technologii OpenCL. 
Technologia tak pozwala na pisanie kodu kt�ry jest przeno�ny mi�dzy wieloma platformami: komputery, urz�dzenia przeno�ne, klastry obliczeniowe. OpenCL pozwala rozprasza� obliczenia na jednostki procesorowe CPU oraz na architektury graficzne GPU. Bardzo wa�n� zalet� OpenCL jest to, �e pisanie z u�yciem tej technologii nie jest zale�ne od sprz�tu na jakim b�dzie ona uruchamiana.

\subsection{ATI Stream Computing}
Technologia ta zosta�a stworzona przez firm� AMD. Za pomoc� tej platformy jeste�my w stanie przeprowadza� z�o�one obliczenia na sprz�cie produkowanym przez AMD.
W sk��d ca�ego pakietu ATI Stream Computing wchodzi autorski j�zyk ATI Brook+ i kompilator tego� j�zyka. Dodatkowo ATI wspiera developer�w w�asn� biblioteka matematyczna (AMD Core Math Library) oraz narz�dziami do profilowania wydajno�ci kodu (Strem Kernel Analyzer). Technologia ATI konkuruje od dawna z technologi� NVIDIA CUDA.


\subsection{NVIDIA CUDA}
CUDA (Compute Unified Device Architecture) jest technologia opracowan� przez firm� NVIDIA. Swoje pocz�tki CUDA mia�a w 2007 roku i do dzi� jest wiod�c� technologi� strumieniowego przetwarzania danych z wykorzystaniem uk�ad�w graficznych GPU. Dalszemu opisu niniejszej technologii po�wi�cony zostanie osobny rozdzia�.

\begin{figure}[h]
\begin{center}
\begin{minipage}[b]{4cm}
\centering
\includegraphics[width=\textwidth]{roz1/img/cuda_logo.jpg}\\\textit{a) Logo NVIDIA CUDA}
\end{minipage}
\begin{minipage}[b]{4cm}
\centering
\includegraphics[width=\textwidth]{roz1/img/ati_logo.jpg}\\\textit{b) Logo ATI Stream Computing}
\end{minipage}
\caption{Loga dw�ch konkurencyjnych ze sob� technologii przetwarzania r�wnoleg�ego na kartach graficznych}
\label{fig:cuda_ati}
\end{center}
\end{figure}


\clearpage

% ********** Rozdzia� 2 **********
\chapter{Cele pracy}
\label{sec:chapter2}


\section{Opracowanie techniki zr�wnoleglenia i przyspieszenia metody �ledzenia promieni przy u�yciu \\NVIDIA CUDA}
Celem niniejszej pracy jest przeniesienie a zarazem zr�wnoleglenie algorytmu �ledzenia promieni na procesory graficzne (GPU) firmy NVIDIA. Celem tak�e jest przy�pieszenie oblicze� standardowego wstecznego Raytracingu w celu jak najszybszego generowania obraz�w scen 3D.


\section{Projekt uniwersalnej aplikacji - benchmark}
W ramach projektu napisany zosta� uniwersalny system Raytracingu dzia�aj�cy na wielordzeniowych procesorach komputerowych (CPU), a tak�e na kartach graficznych (GPU) firmy NVIDIA kt�re obs�uguj� technologie NVIDIA CUDA. Aplikacja testowa jest benchmarkiem, kt�ry jest w stanie przetestowa� zadane sceny 3D na wielu r�nych konfiguracjach sprz�towych. Aplikacja ma za zadanie po uruchomieniu na komputerze u�ytkownika, testowa� wszelkie sceny z odpowiedniego katalogu. Dodatkowo zbiera� potrzebne informacje o sprz�cie u�ytkownika oraz czasy generowania obraz�w z ka�dej ze scen. Po przeprowadzeniu wszelkich test�w aplikacja jest w stanie wys�a� na adres e-mail developera (w tym przypadku autora pracy) wszelkie zgromadzone dane.

% ********** Rozdzia� 3 **********
\chapter{Wprowadzenie do Raytracingu}
\label{sec:chapter3}


\section{Wst�pny opis}
\label{sec:chapter3:Wstep}
W rzeczywistym �wiecie promienie �wietlne rozchodz� si� od �r�d�a �wiat�a do obiekt�w znajduj�cych si� w �wiecie. Ka�de �r�d�o �wiat�a wysy�a niesko�czon� liczb� swoich promieni �wietlnych. Nast�pnie te promienie odbijaj�c si� od obiekt�w i trafiaj� do oczu obserwatora powoduj�c, �e widzi on okre�lony kolor danego obiektu. Gdyby zaadaptowa� t� metod� do generowania realistycznej grafiki komputerowej, otrzymaliby�my dok�adny \\i realistyczny obraz. 
Jednak z racji tego, �e sprz�t komputerowy ma ograniczone mo�liwo�ci, a metoda ta jest bardzo nieefektywn� metod� pod wzgl�dem obliczeniowym. Najszerzej stosowan� metod� �ledzenia promieni jest wsteczne �ledzenie promieni (backward raytracing). W odr�nieniu od post�powego algorytmu �ledzenia promieni (forward raytrcing), kt�re opiera si� na generowaniu jak najwi�kszej liczby promieni dla ka�dego �r�d�a �wiat�a. Algorytm wstecznego �ledzenia promieni zak�ada, �e promienie �ledzone s� od obserwatora, poprzez scen� do obiekt�w z kt�rymi koliduj�. Na rysunku \ref{fig:barwa_pixela} przedstawiony jest pogl�dowy schemat �ledzenia pojedynczego promienia od obserwatora poprzez okre�lony piksel na ekranie

\begin{figure}[h]
	\centering
		\includegraphics[width=0.5\textwidth]{roz3/img/barwa_pixela.png}
	\caption{Spos�b okre�lania barwy piksela w raytracigu\newline\cite{rt_wiki} \url{"http://pl.wikipedia.org/wiki/Ray_tracing"}}
	\label{fig:barwa_pixela}
\end{figure}
\begin{figure}[h]
	\centering
		\includegraphics[width=0.5\textwidth]{roz3/img/rekursywny_algorytm.png}
	\caption{Zasada dzia�ania rekursywnego algorytmu raytracingu \newline\cite{rt_wiki} \url{"http://pl.wikipedia.org/wiki/Ray_tracing"}}
	\label{fig:rekursywny_algorytm}
\end{figure}


\section{Rekursywna metoda �ledzenia promieni}
\label{sec:chapter3:RekursywnaMetoda}
Przy omawianiu wstecznej metody �ledzenia promieni warto wspomnie� o raytracingu rekursywnym. W zagadnieniu tym bada si� rekurencyjnie promienie odbite zwierciadlane oraz za�amane, kt�re powsta�y z kolizji promieni pierwotnych 
z obiektami na scenie. Tak wi�c �ywotno�� promienia pierwotnego wcale nie ko�czy si� w momencie kolizji z obiektem sceny. To czy z danego promienia pierwotnego wygenerowane zostan� kolejne promienie w bardzo du�ej mierze zale�y od materia�u jakim pokryty jest dany obiekt sceny. Z pomoc� tej rekursywnej metody �ledzenia promieni jeste�my w stanie zasymulowa� obiekty lustrzane oraz obiekty p�przezroczyste. Rekurencja w tej metodzie trwa do osi�gni�cia maksymalnego stopnia zag��bienia. Kolor wynikowy danego pojedynczego piksela powstaje z sumy kolor�w, obiektu w jaki trafi� promie� pierwotny oraz kolor�w obiekt�w \\w jakie trafi�y promienie wt�rne. Na rysunku \ref{fig:rekursywny_algorytm} przedstawiony jest pogl�dowy schemat zasady dzia�ania rekursywnej metody �ledzenia promieni.




\section{Przedstawienie algorytmu �ledzenia promieni}
\label{sec:chapter3:PrzedstawienieAlgorytmu}
�ledzenie promieni przez scen� rozpoczyna si� od obserwatora okre�lanego cz�sto jako kamery wyst�puj�cej na scenie. Przez ka�dy piksel ekranu �ledzone s� promienie kt�re poruszaj� si� po scenie. Gdy kt�ry� ze �ledzonych promieni napotka obiekt i zacznie z nim kolidowa�, wtedy z takiego promienia pierwotnego generowane s� promienie wt�rne odbite i za�amane, oczywi�cie w zale�no�ci od materia�u jakim pokryty jest obiekt.


Poni�ej przedstawiony jest schematyczny przebieg algorytmu wstecznego �ledzenia promieni:
\begin{lstlisting}[language=C,style=outcode]

�led� promienie pierwotne

Sprawd� kolizje ze wszystkimi obiektami

Kolor piksela = kolor otoczenia

LOOP( Dla ka�dego zr�d�a �wiat�a ) {
	�led� promie� cienia
	Kolor piksela = wsp�czynnik cienia * kolor obiektu w kt�ry
													  trafi� promie�
}

IF( Obiekt ma w�a�ciwo�ci odbijaj�ce ) {
	kolor piksela += wsp�czynnik odbicia * �led� promie� 
															    odbity
}
IF( Obiekt ma w�a�ciwo�ci za�amuj�ce ) {
	kolor piksela += wsp�czynnik za�amania * �led� promie� 
																	za�amany
}
\end{lstlisting}


\section{Spos�b zr�wnoleglenia algorytmu �ledzenia \\promieni}
\label{sec:chapter3:SposobZrownolelenia}
Z racji tego, �e w standardowym wstecznym algorytmie �ledzenia promieni, promienie przechodz�ce przez poszczeg�lne piksele ekranu nie s� od siebie zale�ne, jeste�my w stanie dokonywa� na nich r�wnoleg�ych oblicze�. W nieniejszej pracy uwaga skupiona zosta�a na technologi� NVIDIA CUDA i to w�a�nie za pomoc� niej jeste�my w stanie dokona� takich r�wnoleg�ych oblicze�. Potrzebne do tego jest przeniesienie algorytmu �ledzenia promieni z wersji CPU, gdzie zazwyczaj odbywa si� ona w spos�b iteracyjny, poprzez przechodzenie w p�tli kolejnych pikseli ekranu. Tak przeniesiony algorytm bedzie wykonywany na ka�dym z w�tk�w udost�pnionych przez CUDA dla danej karty graficznej. Wszelkie obliczenia b�d� wykonywa�y si� r�wnolegle.









% ********** Rozdzia� 4 **********
\chapter{NVIDIA CUDA jako znakomita platforma do zr�wnoleglenia oblicze�}
\label{sec:chapter4}


\section{Wst�pny opis}
\label{sec:chapter4:Wstep}
CUDA(Compute Unified Device Architecture) jest do�� now� technologi� wprowadzon� na rynek przez firm� NVIDIA. Technologia ta sw�j pocz�tek mia�a w 2007 roku. Od samego pocz�tku sta�a si� ona wiod�c� technologi� przetwarzania strumieniowego z wykorzystaniem GPU. CUDA jako, �e jest technologi� stworzon� przez firm� NVIDIA, wspierana jest przez uk�ady graficzne w�a�nie tej firmy. Wsparcie dla tej technologii rozpocz�o si� od uk�ad�w graficznych serii GeForce 8, Quadro oraz Tesla. Seria uk�ad�w graficzny Quadro oraz Tesla s� wyspecjalizowanymi uk�adami obliczeniowymi do zastosowa� naukowych. Natomiast serie GeForce mo�na spotka� na co dzie� w komputerach stacjonarnych oraz laptopach. Z pomoc� technologii CUDA jeste�my wstanie uzyska� wielokrotne przy�pieszenie w obliczeniach w stosunku do oblicze� na zwyk�ym procesorze CPU. na ryskunku \ref{fig:processing_flow_cuda} przedstawiony zosta� przyk�adowy schemat przep�ywu oblicze� w CUDA.

\begin{figure}[h]
	\centering
		\includegraphics[width=0.5\textwidth]{roz4/img/processing_flow_cuda.png}
	\caption{Przyk�ad przep�ywu przetwarzania w technologii CUDA.}
	\label{fig:processing_flow_cuda}
\end{figure}

\section{Wspierane karty oraz zdolno�� obliczeniowa}
\label{sec:chapter4:Wspierane_karty}
We wst�pnym opisie powiedziane by�o, �e technologia CUDA zapocz�tkowana by�a w uk�adach graficznych serii GeForce, Tesla oraz Quadro. W tabeli \ref{tab:wspierane_karty} przedstawione zosta�o oficjalne wsparcie okre�lonej wersji CUDA w poszczeg�lnych uk�adach graficznych.
\\\\\\\\

\begin{table}[h]
\centering
\begin{tabular}{| p{5cm} | p{2cm} | p{6cm} |}
\hline
Zdolno�� obliczeniowa (wersja) & GPUs & Cards \\
\hline
1.0 & G80 & GeForce 8800GTX/Ultra/GTS, Tesla C/D/S870, FX4/5600, 360M \\ \hline
1.1 & G86, G84, G98, G96, G96b, G94, G94b, G92, G92b & GeForce 8400GS/GT, 8600GT/GTS, 8800GT, 9600GT/GSO, 9800GT/GTX/GX2, GTS 250, GT 120/30, FX 4/570, 3/580, 17/18/3700, 4700x2, 1xxM, 32/370M, 3/5/770M, 16/17/27/28/36/37/3800M, NVS420/50 \\ \hline
1.2 & GT218, GT216, GT215 & GeForce 210, GT 220/40, FX380 LP, 1800M, 370/380M, NVS 2/3100M \\ \hline
1.3 & GT200, GT200b & GTX 260/75/80/85, 295, Tesla C/M1060, S1070, CX, FX 3/4/5800 \\ \hline
2.0 & GF100, GF110 & GTX 465, 470/80, Tesla C2050/70, S/M2050/70, Quadro 600,4/5/6000, Plex7000, 500M, GTX570, GTX580 \\ \hline
2.1 & GF108, GF106, GF104 & GT 420/30/40, GTS 450, GTX 460 \\ \hline
\end{tabular}
\caption{Zestawienie kart graficznych oficjalnie wspieraj�cych technologi� CUDA.}
\label{tab:wspierane_karty}
\end{table}

Kolejn� wa�n� rzecz� wyr�niaj�ca karty graficzne jest ich zdolno�� obliczeniowa (ang. compute capability). Identyfikuje ona mo�liwo�ci obliczeniowe danej karty graficznej w odniesieniu do technologii NVIDIA CUDA.  W tabeli \ref{tab:porownanie_zdolnosci} przedstawione zosta�y mo�liwo�ci kart graficznych w zale�no�ci od profilu CUDA.


\begin{table}[h]
\centering
\begin{tabular}{ | p{6cm} | p{1cm} | p{1cm} |  p{1cm} |  p{1cm} |}
\hline
Zdolno�� obliczeniowa & 1.0 & 1.1 & 1.2 & 1.3 \\
\hline
Funkcje atomowe w pami�ci globalnej & - & \checkmark & \checkmark & \checkmark \\ \hline
Funkcje atomowe w pami�ci wsp�dzielonej & - & - & \checkmark & \checkmark \\ \hline
Ilo�� rejestr�w na multiprocesor & 8192 & 8192 & 16384 & 16384 \\ \hline
Maksymalna liczba warp�w na multiprocesor & 24 & 24 & 32 & 32 \\ \hline
Maksymalna liczba aktywnych w�tk�w na multiprocesor & 768 & 768 & 1024 & 1024 \\ \hline
Podw�jna precyzja & - & - & - & \checkmark \\ \hline
\end{tabular}
\caption{Por�wnanie zdolno�ci obliczeniowych kart graficznych wspieraj�cych NVIDIA CUDA.}
\label{tab:porownanie_zdolnosci}
\end{table}

\section{Architektura}
\label{sec:chapter4:architektura}
Karty graficzne GPU znacznie r�ni� si� wydajno�ci� od zwyk�ych procesor�w CPU. R�nica w wydajno�ci wynika g��wnie z faktu, i� procesory graficzne specjalizuj� si� w r�wnoleg�ych, wysoce intensywnych obliczeniach. Karty graficzne sk�adaj� si� z wi�kszej liczby tranzystor�w kt�re s� odpowiedzialne za obliczenia na danych. Nie posiadaj� natomiast takiej kontroli przep�ywu instrukcji oraz jednostek odpowiedzialnych za buforowanie danych jak procesory komputerowe CPU.  Uk�ady graficzne wspieraj�ce technologi� CUDA zbudowane z multiprocesor�w strumieniowych (ang. stream multiprocessor). R�ne modele kart graficznych firmy NVIDIA posiadaj� r�n� liczne multiprocesor�w, co przek�ada si� tak�e na wydajno�� i zdolnos� obliczeniow� danej architektury. Na rysunku \ref{fig:multiprocesor}
Przedstawiona jest przyk�adowa budowa takiego w�a�nie multiprocesora.

\begin{figure}[h]
	\centering
		\includegraphics[width=0.5\textwidth]{roz4/img/multiprocesor.jpg}
	\caption{Przyk�adowy schemat multiprocesora strumieniowego.}
	\label{fig:multiprocesor}
\end{figure}


Ka�dy z multiprocesor�w sk�ada si� z: (napisac z kad to wziete!!!!!)
\begin{itemize}  
  \item I-Cache - bufor instrukcji  
  \item MT Issue - jednostka kt�ra rozdziela zadania dla SP i SFU
  \item C-Cache -bufor sta�ych (ang. constant memory) o wielko�ci 8KB, kt�ry przyspiesza odczyt z obszaru pami�ci sta�ej 
  \item 8 x SP - 8 jednostek obliczeniowych tzw stream processors, kt�re wykonuj� wi�kszo�� oblicze� pojedynczej precyzji (ka�dy zawiera w�asne 32-bitowe rejestry)
  \item 2 x SFU  - jednostki specjalne (ang. special function units). Zadaniem ich jest obliczanie funkcji przest�pnych, np. trygonometrycznych, wyk�adniczych i logarytmicznych, czy interpolacja parametr�w. 
  \item DP -procesor, kt�ry wykonuje obliczenia podw�jnej precyzji
  \item SM - pami�� wsp�dzielona (ang. shared memory) o wielko�ci  16KB.
\end{itemize}  

\section{Rodzaje pami�ci w architekturze CUDA}
\label{sec:chapter4:pamieci}
\begin{itemize} 
\item Pami�� globalna (ang. global memory) - Ta pami�� jest dost�pna dla wszystkich w�tk�w. Nie jest pami�ci� buforowan�. Dost�p do niej trwa od oko�o 400 do 600 cykli. Pami�� ta s�u�y przede wszystkim do zapisuj wynik�w dzia�a� programu obliczeniowego.

\item Pami�� lokalna (ang. local memory) - Ma taki sam czas dost�pu jak pami�� globalna (400-600 cykli). Nie jest tak�e pami�ci� buforowan�. Jest ona zdefiniowana dla danego w�tku. Ka�dy w�tek CUDA posiada w�asn� pami�� lokaln�. Zajmuje si� ona przechowywaniem bardzo du�ych struktur danych. Pami�� ta jest najcz�ciej u�ywana gdy obliczenia danego w�tku nie mog� by� w ca�o�ci wykonane na dost�pnych rejestrach procesora graficznego.

\item Pami�� wsp�dzielona (ang. shared memory) - Jest to bardzo szybki rodzaj pami�ci, dor�wnuj�cy szybko�ci rejestr�w procesora graficznego. Przy pomocy tej pami�ci, w�tki przydzielone do jednego bloku s� wstanie si� ze sob� komunikowa�. Nale�y jednak obchodzi� si� ostro�nie z tym rodzajem pami�ci, gdy� mog� powsta� Momoty w kt�rych w�tki w jednym bloku b�d� chcia�y jednocze�nie zapisywa� i odczytywa� z tej pami�ci. Wyst�powanie takich konflikt�w w odczycie i zapisie powoduje du�e op�nienia.

\item Pami�� sta�a (ang. const memory) - Ta pami�� w odr�nieniu do powy�szych rodzaj�w pami�ci, jest buforowan� pami�ci� tylko do odczytu. Gdy potrzebne dane znajduj� si� aktualnie w buforze dost�p do nich jest bardzo szybki. Czas dost�pu ro�nie gdy danych nie ma w buforze i musz� by� doczytane z pami�ci karty.

\item Pami�� Tekstur (ang. texture memory) - Jest pami�ci� podobn� do pami�ci sta�ej gdy� udost�pnia tylko odczyt danych. Jest tak�e pami�ci� buforowan�. W pami�ci tej bufor danych zosta� zoptymalizowany pod k�tek odczytu danych z bliskich sobie adres�w. Najkorzystniejsz� sytuacj� jest gdy w�tki dla danego warpa (grupa 32 w�tk�w zarz�dzanych przez pojedynczy multiprocesor) odczytuj� adresy, kt�re znajduj� si� blisko siebie. CUDA w swojej implementacji udost�pnia mo�liwo�� pos�ugiwania si� teksturami 1D,2D,3D.

\item Rejestry - Jest to najszybszy rodzaj pami�ci. Dost�p do niego nie powoduje �adnych dodatkowych op�nie�, chyba �e pr�bujemy odczyta� z rejestru do kt�rego dopiero co zosta�o co� zapisane. Ka�dy multiprocesor w urz�dzeniu CUDA posiada 8192 lub 16384 rejestr�w 32-bitowych. Zale�y to od wersji(zdolno�ci obliczeniowej) danego urz�dzenia. W celu unikni�cia powy�szych konflikt�w ilo�� w�tk�w na pojedynczy multiprocesor ustawia si� jako wielokrotno�� liczby 64. NAPISAC Z KAD TO WIEM!!!!!!!!!!!!!1
\end{itemize} 


Na obrazku \ref{fig:pamiec} poni�ej przedstawiony zosta� pogl�dowy schemat pami�ci w architekturze CUDA.
\begin{figure}[h]
	\centering
		\includegraphics[width=0.5\textwidth]{roz4/img/pamiec.jpg}
	\caption{Schemat pami�ci.}
	\label{fig:pamiec}
\end{figure}



\section{Przyk�adowy program pod architektur� CUDA}
\label{sec:chapter4:kod}
Poni�ej przedstawiony zosta� przyk�ad programu napisanego w j�zyku C dla architektury CUDA. Program ten uruchamiany jest na wielu w�tkach karty graficznej, ka�dy z tych w�tk�w niezale�nie wpisuj� do tablicy swoje ID.
Wa�n� informacj� przy pisaniu kodu dla architektury CUDA jest to, �e funkcje uruchamiane przez w�tki maj� specjalne oznaczenia:
\begin{itemize} 
\item global - funkcje tak� wywo�a� mo�na tylko z CPU, a wykonuje si� ona na GPU
\item host - funkcjawykonuje si� i mo�e by� wywo�ana tylko z kodu wykonywanego na CPU
\item device - funkcja wykonuje si� i mo�e by� wywo�ana tylko z kodu wykonywanego na GPU
\end{itemize} 

Nale�y tak�e pami�ta� �e funkcje dla w�tk�w CUDA musz� zawsze zwracac wartosc \textit void.

\begin{lstlisting}[language=C,style=outcode]
#include <stdlib.h>
#include <cuda_runtime.h>
#include <cutil.h>

// definicja funkcji kt�ra b�dzie uruchamiana 
// r�wnolegle na w�tkach CUDA
__global__ void testFunction(int *data)
{
	// obliczamy index tablicy a zarazem w�tku
	int id = blockIdx.x * blockDim.x + threadIdx.x;
	// Zaposujemy do tablicy ID w�tku
	data[id] = id;
}

// W funkcji main wywo�ujemy powy�sz� funkcje
// dla w�tk�w CUDA
int main()
{
	// Na poczatku nale�ey zainicjowa� urz�dzenie CUDA
	cudaSetDevice(0);

	// alokujemy pamie� na karcie graficznej
	int *tablica;
	cudaMalloc((void**)&tablica, sizeof(int) * ARRAY_SIZE); 

	// Ustalamy wielkosc bloku i karty
	dim3 dimBlock(BLOCK_SIZE, 1); 
	dim3 dimGrid(ARRAY_SIZE / dimBlock.x, 1);

	// wywo�ujemy nasz� funkcj� obliczeniow�
	testFunction<<<dimGrid, dimBlock>>>(tablica);

	// Tworzymy tablice w pamieci ram i kopujemy
	// dane z karty graficznej do pamieci ram.
	int *tablica2 = (int*)malloc(sizeof(int) * ARRAY_SIZE);
	cudaMemcpy(tablica, tablica2, sizeof(int) * ARRAY_SIZE,
	cudaMemcpyDeviceToHost);

	return 0;
}
\end{lstlisting}

Jak widzimy na powy�szym listingu kodu gdy wywo�ujemy funkcj� CUDA okre�lamy na ilu w�tkach ma si� ona uruchomi� i w jakie grupy maj� by� one pogrupowane.
Na rysunku JAKIS RYSSSSSSSSSSSSSSSss przedstawiony zosta� schemat pokazuj�cy jak mog� wygl�da� u�ywane w�tki w ca�ej kgracie, pogrupowane w odpowiednie bloki.
Podczas programowania na karty graficzne CUDA nale�y pami�ta� o r�nych dost�pnych rodzajach pami�ci i wybra� t� naj�a�ciwsz�. Je�li nie przemy�limy dobrze problemu jaki sobie za�o�yli�my rozwi�za� przy pomocy technologii CUDA, mo�e si� zda�y�, �e nasze rozwi�zanie b�dzie dzia�a�o gorej ni� na procesorze CPU. Nale�y tak�e poinformowa� o tym, �e brak jest narz�dzi, kt�re wspomaga�y by �ledzenie przep�ywu wykonywania programu tzw debugowanie. Z tym problemem borykaja si� wszystkie technologie zwiazane z GPGPU( obliczenia przeprowadzane na kartach graficznych ).

% ********** Rozdzia� 5 **********
\chapter{Om�wienie aplikacji testowej}
\label{sec:chapter5}


\section{Za�o�enia}
\label{sec:chapter5:zalozenia}
Na potrzeby niniejszej pracy zosta�o opracowane autorskie rozwi�zanie uniwersalnego wstecznego raytracera dzia�aj�cego zar�wno na procesorze CPU jak i r�wnie� na ko-procesorach graficznych GPU firmy NVIDIA. Aplikacja testowa jest wstanie generowa� wynikowe obrazy scen 3D sk�adaj�cych si� z kul, prostopad�o�cian�w oraz p�aszczyzn. Na ka�dy z element�w sceny jest mo�liwo�� na�o�enia dowolnej tekstury oraz doboru odpowiednich parametr�w materia�u. Dodatkowo na scenie mo�liwe jest umieszczanie �wiate� punktowych. Aplikacja sama w sobie jest benchmarkiem, kt�ry potrafi przetestowa� zadan� liczb� scen 3D na komputerze u�ytkownika. Zebrane wyniki z oblicze� jest wstanie przes�a� na wybrany adres e-mail (w tym przypadku za zgod� u�ytkownika do developera). Aplikacja przy generowaniu obrazu sceny 3D bierze pod uwagi r�ne w�a�ciwo�ci materia�u danego obiektu. Docelowo generowane s� takie efekty jak: o�wietlenie, odblask, cienie, wielokrotne odbicia i za�amania, tekstury. Przy u�yciu materia��w o r�nych parametrach jeste�my wstanie uzyska� bardzo ciekawie wygl�daj�ce obiekty np: lustro, szk�o, metale i wiele innych.


\section{Implementacja}
\label{sec:chapter5:implementacja}
Aplikacja testowa zosta�a napisana w j�zyku C++, wykorzystuj�c biblioteki standardowe pochodz�ce z j�zyka C. Wersja �ledzenia promieni przy u�yciu technologii CUDA zosta�a napisana w tzw. "C for CUDA". Dodatkowo do wy�wietlania wynikowych obraz�w u�yta zosta�a biblioteka Microsoft DirectX 9.0. Program przeznaczony jest do uruchamiania na systemach z rodziny Windows. Aplikacj� testow� mo�na nazwa� swoistym benchmarkiem. Dzia�anie jej sk�ada si� z 5 wa�nych punkt�w:
\begin{itemize}  
\item wczytywanie scen do testow
\item testowanie zadanych scen na procesorze CPU.
\item testowanie zadanych scen na karcie graficznej GPU.
\item zapisywanie wynikowych obraz�w scen na dysk u�ytkownika
\item zebranie informacji o testowanych scenach i wys�anie ich na mail developera.
\end{itemize}

\textbf{Przebieg dzia�ania:}\\
Aplikacja na samym pocz�tku wczytuje plik benchamarku z rozszerzeniem *.rtb. Plik ten zawiera w sobie spis scen (pliki *.rtm) kt�re maj� by� przetestowane przez raytracer. Nast�pnie rozpoczyna si� testowanie zadanych scen na procesorze CPU.  Gdy wszystkie sceny zostan� przetestowane na procesorze, rozpoczyta si� raytracing na karcie graficznej z u�yciem CUDA. Na koniec gdy ju� wszystkie sceny zosta�y wygenerowane przy u�yciu CPU oraz GPU, wynikowe obrazy generowane przez raytracer zapisywane s� na dysku u�ytkownika. Zebrane informacje z profilowania ka�dej ze scen zostaj� zapisane do pliku oraz wys�ane na adres e-mail developera.
\\\\
\textbf{Statystyki zwi�zane z kodem aplikacji testowej:}\\
\begin{itemize} 
\item 61 plik�w kodu
\item 9386 linii kodu
\item 272697 bajt�w kodu
\end{itemize} 


\section{Zestaw test�w}
\label{sec:chapter5:zestaw_testow}
By wykaza� przy�pieszenie pomi�dzy �ledzeniem promieni na procesorze CPU a kart� graficzn� GPU przygotowany zosta� zestaw 10 scen testowych. Testowana jest wydajno�� generowania scen o r�nej budowie i wyst�puj�cych na niej prymitywach. W scenach tych testowanych jest wiele parametr�w takich jak:
\begin{itemize} 
\item Odbicia promieni od obiekt�w na scenie
\item Za�amania promieni w obiektach na scenie
\item Teksturowanie obiekt�w sceny
\item Rodzaj oraz ilo�� prymityw�w wy�wietlanych na scenie
\item Jako�� generowanego obraz (super sampling)
\item Rozdzielczo�� generowanego obrazu
\end{itemize} 


\section{Przyk�ady wygenerowanych obraz�w}
\label{sec:chapter5:wygenerowane_obrazy}
W rozdziale tym przedstawione zosta�y wyniki generowania scen przez aplikacje testow�. Ka�da z tych scen by�a generowana na procesorze CPU oraz na karcie graficznej GPU.
\begin{figure}[h]
	\centering
		\includegraphics[width=0.5\textwidth]{roz5/img/rt1.png}
	\caption{Przyk�adowa wygenerowana scena 1.}
	\label{fig:rt1}
\end{figure}
\begin{figure}[h]
	\centering
		\includegraphics[width=0.5\textwidth]{roz5/img/rt2.png}
	\caption{Przyk�adowa wygenerowana scena 2.}
	\label{fig:rt2}
\end{figure}
\begin{figure}[h]
	\centering
		\includegraphics[width=0.5\textwidth]{roz5/img/rt3.png}
	\caption{Przyk�adowa wygenerowana scena 3.}
	\label{fig:rt3}
\end{figure}

% ********** Rozdzia� 6 **********
\chapter{Por�wnanie wydajno�ci Raytracera dzia�aj�cego na CPU oraz na GPU}
\label{sec:chapter6}
Tutaj przedstawi� wszelkie wyniki generowania obraz�w ze scen oraz wszelkie warto�ci czasowe im odpowiadaj�ce.  Por�wnanie wynik�w dla raytracingu CPU vs GPU. Z ka�dych wynikowych danych bada� przeprowadzonych na r�nych konfiguracjach sprz�towych wygenerowany zostanie wykres przedstawiaj�cy r�nice czasowe w generowaniu scen a tym samym przyspieszenie wstecznego raytracingu jakie uda�o si� uzyska�.
Zak�adam wst�pnie testy 10 scen. Na ka�dej z nich r�na konfiguracja obiekt�w oraz materia��w im nadanych. Chce pokaza� jakie w�a�ciwo�ci materia��w i jaka ilo�� prymityw�w oraz ich rodzaj�w wp�ywa na zmniejszanie/zwi�kszanie wydajno�ci raytracingu.

\clearpage
\section{Test1 - Scena z�o�ona z samych sfer}
\label{sec:chapter6:test1sphere}
Testowi poddana zosta�a scena zawieraj�ca jako prymitywy same sfery.
Na scenie jest dok�adnie 20 sfer, kt�re maj� na�o�one r�ne materia�y.
Na scenie tej znajduj� si� tak�e 4 �wiat�a punktowe.

\begin{figure}[h]
	\centering
		\includegraphics[width=0.5\textwidth]{roz6/img/spheres.png}
	\caption{Scena z�o�ona ze sfer}
	\label{fig:test1sphere}
\end{figure}

WYKRES
\clearpage
\section{Test2 - Scena z�o�ona z samych pude�ek}
\label{sec:chapter6:test2boxes}
W tym te�cie na scenie zosta�o umieszczonych 20 pude�ek (box�w). Ka�de z tych pude�ek posiada sw�j w�asny materia�. Scena r�wnie� o�wietlana jest przez 4 �r�d�� �wiat�a punktowego.

\begin{figure}[h]
	\centering
		\includegraphics[width=0.5\textwidth]{roz6/img/boxes.png}
	\caption{Scena z�o�ona ze sfer}
	\label{fig:test1sphere}
\end{figure}

WYKRES
\clearpage
\section{Test3 - Scena z wymieszanymi prymitywami}
\label{sec:chapter6:test2mixed}
Testowi poddana zosta�a scena z r�nego rodzaju prymitywami. Ka�dy z prymityw�w posiada sw�j materia�, niekt�re tak�e teksture. Scena sk��da si� dok�adnie z:
\begin{itemize} 
\item dwudziestu sfer
\item dwudziestu pude�ek
\item sze�ciu p�aszczyzn
\item czterech �r�de� �wiat�a punktowego.
\end{itemize}


\begin{figure}[h]
	\centering
		\includegraphics[width=0.5\textwidth]{roz6/img/mixed_scene.png}
	\caption{Scena z�o�ona ze sfer}
	\label{fig:test1sphere}
\end{figure}

WYKRES

\clearpage
\section{Test2 - R�na rozdzielczo�� generowanych scen}
\label{sec:chapter6:test2resolution}
W te�cie zbadana zostanie zale�no�� szybko�ci generowania sceny w zale�no�ci od rozdzielczo�ci wynikowego obrazu. Testowany jest tutaj spadek wydajno�ci od liczby generowanych promieni.
Przetestowane zosta�y dwa warianty scen:
\begin{itemize}  
\item Rozdzielczo�� obrazu 400 na 400 pikseli.
\item Rozdzielczo�� obrazu 800 na 600 pikseli.
\end{itemize}


OBRAZKI

WYKRESY



\clearpage
\section{Test3 - Ro�ny stopie� dok�adno�ci generowanych scen}
\label{sec:chapter6:test3ss}
W tym te�cie zbadana zosta�a szybko�� wygenerowanego obrazu od jako�ci tego obrazu. Stosowany jest tu tak zwany super-sampling(DODAC DO UZYWANYCH SLOW). Jest to wzmo�one pr�bkowanie promieni na pojedy�czy pixel ekranu.
Przetestowane zosta�y dwa warianty scen:
\begin{itemize}  
\item Scena ze �ledzeniem pojedy�czego promienia na pixel.
\item Scena ze �ledzeniem czterech promieni na pojedy�czy pixel.
\end{itemize}

OBRAZKI

WYKRESY

Jak wida� na za��czonych wykresach, zauwa�alny jest spadek w szybko�ci generowanych scen wraz ze wzrostem �ledzonych promieni na pojedy�czy pixel.
Im mniejszy wspo�czynnik super-samplingu tym szybciej generowana jest scena.
Jednak jej jako�� nie jest najlepsza. Pojawia si� tu zjawisko tzw. aliasingu(DODAC I WYJASNIC NA GORZE w URZYWANYCH OKRESLENIACH). 



\clearpage
\section{Test4 - R�na liczba �wiate� punktowych na scenie}
\label{sec:chapter6:test4lights}
W te�cie sprawdzone zosta�o jaki wp�yw ma liczba �wiate� punktowych na szybko�� generowanej sceny. 
Testy zosta�y wykonane dla czterech r�nych wariant�w:
\begin{itemize}  
\item dwa �r�d�a �wiat�a punktowego.
\item cztery �r�d�a �wiat�a punktowego.
\item osiem �r�de� �wiat�a punktowego.
\item szesna�cie �r�de� �wiat�a punktowego.
\end{itemize}

\begin{figure}[h]
\begin{center}
\begin{minipage}[b]{6cm}
\centering
\includegraphics[width=\textwidth]{roz6/img/light_2.png}\\\textit{a) 2 �r�d�a �wiat�a}
\end{minipage}
\begin{minipage}[b]{6cm}
\centering
\includegraphics[width=\textwidth]{roz6/img/light_4.png}\\\textit{b) 4 �r�d�a �wiat�a}
\end{minipage}
\begin{minipage}[b]{6cm}
\centering
\includegraphics[width=\textwidth]{roz6/img/light_8.png}\\\textit{c) 8 �r�de� �wiat�a}
\end{minipage}
\begin{minipage}[b]{6cm}
\centering
\includegraphics[width=\textwidth]{roz6/img/light_16.png}\\\textit{d) 16 �r�de� �wiat�a}
\end{minipage}
\caption{Obrazy wygenerowane w te�cie o r�nej liczbie �wiate�}
\label{fig:testlights}
\end{center}
\end{figure}

WYKRESY


\section{Test5 - TMP TEXT}
\label{sec:chapter6:test5}
\begin{frame}[fragile]
\begin{lstlisting}
Sample Code
\end{lstlisting}
\end{frame}
\label{sec:chapter6:test6}

\section{Test7 - TMP TEXT}
\label{sec:chapter6:test7}

\section{Test8 - TMP TEXT}
\label{sec:chapter6:test8}

\section{Test9 - TMP TEXT}
\label{sec:chapter6:test9}

\section{Test10 - TMP TEXT}
\label{sec:chapter6:test10}

\section{Podsumowanie test�w}
\label{sec:chapter6:podsumowanie}

% ********** Rozdzia� 7 **********
\chapter{Podsumowanie i wnioski}
\label{sec:chapter7}
W pracy uda�o osi�gn�� si� zamierzone na pocz�tku cele. Zosta�a stworzona uniwersalna aplikacja do testowania wydajno�ci raytraciingu na procesorach komputerowych CPU oraz na kartach graficzych NVIDIA obs�uguj�cych technologi� CUDA. Uzyskano te� znaczne przyspieszenie podczas generowania scen na kartach graficznych. Zas�ug� tej wydajno�ci jest zr�wnoleglenie oblicze� na w�tki obliczeniowe wchodz�ce w sk�ad uk�adu graficznego. Jasno mo�na tutaj powiedzie�, �e w kartach graficznych tkwi wielka pot�ga i obliczenia na nich przeprowadzane przewy�szaj� wydajno�ci� zwyk�e wielordzeniowe procesory komputerowe. Aktualnie, stale utrzymuje si� rosn�cy trend na kolejne lepsze uk�ady graficzne. Wydaje mi si�, �e ju� nied�ugo b�dziemy mogli we w�asnych domowych warunkach uruchamia� z�o�one, dzia�aj�ce w czasie rzeczywistym aplikacje multimedialne, oparte o grafike generowan� metod� raytracingu.


\section{Napotkane problemy}
\label{sec:chapter7:problemy}
Pracuj�c nad niniejsz� prac�, oraz maj�c do czynienia a znow� technoogi� NVIDIA CUDA napotka�em kilka problem�w. Niemniej jednak 3 z nich by�y g��wnymi problemami. Dwa natury softwarowej oraz jeden natury hardwarowej.
\begin{itemize}  
  \item Technologia NVIDIA CUDA nie wspiera rekurencji. W tym wypadku implementacja odbi� i za�ama� �wiat�a nie by�a mo�liwa w standardowy spos�b. Poradzi�em sobie tworz�c sztuczn� rekurencje w p�tli u�ywaj�c w�asnorecznie stworzonego stosu.
  \item Testowanie(debugowanie) algorytm�w pod technologi� NVIDIA CUDA jest mo�liwe tylko i wy��cznie je�li posiada si� sprz�t z dwoma kartami graficznymi wspieraj�cymi w�a�nie t� technologi�. Nie posiada�em takiego sprz�tu wie� moje sprawdzanie poprawni�ci algorytm�w odbywa�o si� metod� pr�b i b��d�w oraz por�wnywaniem wynikowych obraz�w z wersj� raytracera dzia�aj�c� na procesorze CPU.
  \item Ostanim wa�nym problemem, kt�remu musia�em stawi� czo�o by� problem natury sprz�towej. Sprz�t jaki mia�em do dyspozycji posiada� fabrycznie wadliwy uk�ad graficzny (GeForce 8400M). Podczas implementacji raytracingu uk�ad ten przepali� si� oko�o 4 razy i tyle samo razy by� wymieniany w serwisie na nowy.
\end{itemize}


\section{Perspektywy kontynuacji}
\label{sec:chapter7:kontynuacja}
Niniejsz� prac� oraz badania nad metod� �ledzenia promieni chcia�bym dalej rozwija� w ramach kolejnej pracy dyplomowej.
W planach do zrealizowania mam mi�dzy innymi:
\begin{itemize}  
  \item Ulepszenie metody wstecznego raytracingu adaptuj�c algorytmy photon-mappingu oraz path-tracingu
  \item Pr�ba stworzenia w�asnego hybrydowego algorytmu �ledzenia promieni w celu kolejnych przy�piesze� generowania scen.
  \item Przeniesienie raytracingu z CUDA na inne platformy oblicze� strumieniowych: openCL, ATI Stream Computing.
  \item Modyfikacja aplikacji (benchmarku) by by�o mo�liwe uruchomienie jej na innych systemach operacyjnych ni� rodzina Windows.
\end{itemize}



\begin{comment}
\begin{lstlisting}[language=C,style=outcode]

Sledz promienie pierwotne

Sprawdz kolizje ze wszystkimi obiektami

kolor piksela = kolor otoczenia

LOOP( Dla kazdego zr�d�a swiat�a ) {
	Sledz promien cienia
	Kolor piksela = wsp�czynnik cienia * kolor obiektu w kt�ry
													  trafi� promien
}

IF( obiekt ma w�asciwosci odbijajace ) {
	kolor piksela += wsp�czynnik odbicia * sledz promien 
															    odbity
}
ELSE IF( Jezeli obiekt ma w�asciwosci za�amujace ) {
	kolor piksela += wsp�czynnik za�amania * sledz promien 
																	za�amany
}

\end{lstlisting}
\end{comment}






% *************** Bibliography ***************
\nocite{*}
\bibliographystyle{plain}
{\small\bibliography{mobiStopowicz}}

% *************** Appendixes ***************
\appendix
%\appendixpage*
% ********** Dodatek 1 **********
\chapter{Terminologia stosowana w pracy}
\label{sec:appendix1}


\begin{itemize} 
\item CUDA - Technologia stworzona przez firm� NVIDIA w 2007 roku. Umo�liwia r�wnoleg�e obliczenia na mikroprocesorach karty graficznej.
\item Benchmark - Aplikacja testowa, kt�ra profiluje wydajno�� i zbiera informacje.
\item Warp - blok w�tk�w przydzielony na multiprocesor.
\item DirectX - technologia graficzna firmy Microsoft. Umo�liwia wy�wietlanie wysokiej jako�ci grafiki 2D/3D.
\item Aliasing - Zdeformowany, o z�ej jako�ci obraz kt�ry powstaje podczas rastaryzacji, powodowany przez zbyt ma�a cz�stotliwo�� pr�bkowania na pojedy�czy piksel obrazu. Przeciwdzia�a si� temu efektowi poprzez antyaliasing oraz w raytracingu poprzez super-sampling.
\item Super-sampling - spos�b na zwi�kszenie jako�ci generowanych scen. Polega na �ledzeniu wielu promieni �wietlnych na pojedy�czy piksel generowanego obrazu.
\end{itemize} 

% ********** Koniec dodatku **********


% *************** Back matter ***************
\backmatter
\input{back.tex}

\end{document}
