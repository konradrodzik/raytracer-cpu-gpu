% ********** Rozdzia� 2 **********
\chapter{Cele pracy}
\label{sec:chapter2}


\section{Opracowanie techniki zr�wnoleglenia i przyspieszenia metody �ledzenia promieni przy u�yciu \\NVIDIA CUDA}
Celem niniejszej pracy jest przeniesienie a zarazem zr�wnoleglenie algorytmu �ledzenia promieni na procesory graficzne (GPU) firmy NVIDIA. Celem tak�e jest przy�pieszenie oblicze� standardowego wstecznego Raytracingu w celu jak najszybszego generowania obraz�w scen 3D.


\section{Projekt uniwersalnej aplikacji - benchmark}
W ramach projektu napisany zosta� uniwersalny system Raytracingu dzia�aj�cy na wielordzeniowych procesorach komputerowych (CPU), a tak�e na kartach graficznych (GPU) firmy NVIDIA kt�re obs�uguj� technologie NVIDIA CUDA. Aplikacja testowa jest benchmarkiem, kt�ry jest w stanie przetestowa� zadane sceny 3D na wielu r�nych konfiguracjach sprz�towych. Aplikacja ma za zadanie po uruchomieniu na komputerze u�ytkownika, testowa� wszelkie sceny z odpowiedniego katalogu. Dodatkowo zbiera� potrzebne informacje o sprz�cie u�ytkownika oraz czasy generowania obraz�w z ka�dej ze scen. Po przeprowadzeniu wszelkich test�w aplikacja jest w stanie wys�a� na adres e-mail developera (w tym przypadku autora pracy) wszelkie zgromadzone dane.
